\chapter{Formally Defining Sokoban}
\label{cha:strips}
\section{Introduction}
In order to effectively solve the Sokoban problem we need to define
the game in a formal manner. To this end we chose to define the
Sokoban game in a variant of STRIPS.



\section{Analysis}
Numerical STRIPS developed by \citet{Hoffmann03themetric-ff} in
\citeyear{Hoffmann03themetric-ff} adds arithmetic and boolean
operators along with numeric state variables to the STRIPS
language. Using this extension gives us a more powerful language,
which in turn lets us create a more simpel STRIPS model of Sokoban.

We wish to keep our STRIPS description as simple as possible. If we
look at the rules of the game we see that the only real action is to
move the player, everything else happens is a result of this.

% \subsection{Required Actions}
Sokoban can be viewed as a single agent system\footnote{At least in in
  this incarnation. See Section \ref{sec:scope}.} As such the primary
action needed must be to move this agent.

As described in Section \ref{sec:rules} there are a number of
preconditions that need to be satisfied before the move action is
possible (or allowed if you will).

\begin{itemize}
\item The player can move at most a single tile any of the four
  directions.
\item There must not be an obstructing object in the chosen direction.
\end{itemize}

If we create that there is a list of allowed moves, that we do not
alter in any of our action, this is effectively a static list. In our
problem instance this list would contain the values $(x,y)$: $(-1,0)$,
$(1,0)$, $(0,-1)$ and $(0,1)$, representing the directions up, down,
left and right, i.e. we define the upper-left corner to be $(0,0)$,
for implementation purposes. We use the LegalDir predicate to indicate
this notion.


\begin{figure}[!h]
  \centering \stripsaction{Move$(p, x, y, dx, dy)$}%
  {Player$(p)$ $\wedge$ LegalDir$(dx, dy)$ $\wedge$ At$(p, x, y)$
    $\wedge$ Free$(x+dx, y+dy)$}%
  {$\neg$At$(p, x, y)$ $\wedge$ $\neg$Free$(x+dx, y+dy)$ $\wedge$
    Free$(x, y)$ $\wedge$ At$(p, x+dx, y+dy)$}
  \caption{STRIPS description of the Move action. Only move if the
    adjacent spot is free.}
  \label{fig:strips-move}
\end{figure}


Figure \ref{fig:strips-move} show our STRIPS description of the Move
action. The Free predicate indicates whether a given space is
unoccupied. After the move we mark the source as being Free and the
destination as being occupied.

\begin{figure}[!h]
  \centering \stripsaction{Push$(p, b, x, y, dx, dy)$}%
  {Player$(p)$ $\wedge$ LegalDir$(dx, dy)$ $\wedge$ Box$(b)$ $\wedge$
    At$(p, x, y)$ $\wedge$ At$(b, x+dx, y+dy)$ $\wedge$ Free$(x+dx+dx,
    y+dy+dy)$ $\wedge$ $\neg$Target$(x+dx+dx, y+dy+dy)$}%
  {$\neg$At$(p, x, y)$ $\wedge$ $\neg$At$(b, x+dx, y+dy)$ $\wedge$
    $\neg$Free$(x+dx+dx, y+dy+dy)$ $\wedge$ Free$(x, y)$ $\wedge$
    At$(p, x+dx, y+dy)$ $\wedge$ At$(b, x+dx+dx, y+dy+dy)$ $\wedge$
    $\neg$AtTarget$(b)$}
  \caption{STRIPS description of the Push action}
  \label{fig:strips-push}
\end{figure}

In order to move the boxes we must be adjacent to them and the box
present in the given direction must have free space on the other side
of it. Otherwise the requirements are them same is it is to the Move
action. We refer to Figure \ref{fig:strips-push}.

\begin{figure}[!h]
  \centering \stripsaction{PushToTarget$(p, b, x, y, dx, dy)$}%
  {Player$(p)$ $\wedge$ LegalDir$(dx, dy)$ $\wedge$ Box$(b)$ $\wedge$
    At$(p, x, y)$ $\wedge$ At$(b, x+dx, y+dy)$ $\wedge$ Free$(x+dx+dx,
    y+dy+dy)$ $\wedge$ Target$(x+dx+dx, y+dy+dy)$}%
  {$\neg$At$(p, x, y)$ $\wedge$ $\neg$At$(b, x+dx, y+dy)$ $\wedge$
    $\neg$Free$(x+dx+dx, y+dy+dy)$ $\wedge$ Free$(x, y)$ $\wedge$
    At$(p, x+dx, y+dy)$ $\wedge$ At$(b, x+dx+dx, y+dy+dy)$ $\wedge$
    AtTarget$(b)$}
  \caption{STRIPS description of the Push to target action}
  \label{fig:strips-pushtotarget}
\end{figure}

The last push, onto a goal square, is somewhat special. We need a
seperate action to mark that the box in question is now on a goal
square. Since this is the only action that can push a box onto the
goal square (due to preconditions in both Move and Push) we know that,
using a regular push action will never result in box landing on a goal
square.


\section{Summary}
Using the STRIPS language we have created a very small description of
our system. Having this very compact description will enable us to
build a highly effective model and solver, since we are able to keep a
system-wide overview.




