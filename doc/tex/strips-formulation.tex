\chapter{Formally Defining Sokoban}

\section{Introduction}
In order to effectively solve the Sokoban problem we need to define
the game in a formal manner. To this end we chose to define the
Sokoban game in a variant of STRIPS.



\section{Analysis}
Numerical STRIPS developed by \citet{Hoffmann03themetric-ff} in
\citeyear{Hoffmann03themetric-ff} adds arithmetic and boolean
operators along with numeric state variables to the STRIPS
language. Using this extension gives us a more powerful language,
which in turn lets us create a more simpel STRIPS model of Sokoban.

We wish to keep our STRIPS description as simple as possible. If we
look at the rules of the game we see that the only real action is to
move the player, everything else happens is a result of this.

\subsection{Required Actions}

% Rolling Stone 
% http://www.cs.ualberta.ca/~games/Sokoban/ 


\stripsaction{a}{b}{c}
%\stripsaction{$Move(p, x, y, dx, dy)$}%
%{$Player(p) \wedge LegalDir(dx, dy) \wedge At(p, x, y) \wedge Free(x+dx, y+dy)$}{test}
%{$\neg~At(p, x, y) \wedge \neg~Free(x+dx, y+dy) \wedge Free(x, y) \wedge At(p, x+dx, y+dy)$}




\subsection{What we need to know about the world}



