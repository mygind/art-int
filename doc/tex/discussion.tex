
\chapter{Discussion}
\label{cha:discussion}

\section{Observation}
Difference between levels that are easy for humans and computers.

\section{Improvements}
\section{More deadlock detection}
To improve the algorithm, better deadlock detections must be
created. The first to implement would be heuristics that detect the
states in figure \ref{fig:missingdeadlocks}.

First of all it should not be allowed to put two boxes into the same
narrow corridor as you cannot get in between the boxes thus making the
boxes deadlock. (The left image in figure \ref{fig:missingdeadlocks}.)
One way of solving this could be to make the corridor a one box only
zone and lock it as soon as a box is inside it.

Furthermore a deadlock detector that is able to check if a room with
only one entrance is getting blocked, must be created. (The right
image). There are many aspects in this: first of all how do you find
areas with only one entrance? And how do you decide whther the
entrance is blocked? And how do you do this without making the
algorithm very slow?

\begin{figure}[htp]
  \centering
  \includegraphics[width=0.4\textwidth]{blockednarrow}
  \includegraphics[width=0.4\textwidth]{blockedentrance}
  \caption{Deadlocks not found by our heuristics}
  \label{fig:missingdeadlocks}
\end{figure}


POP planning, regression planning?

Variants of sokoban
\begin{itemize}
\item Colored boxes
\item Several Pushers
\item Push rows of boxes
\item Traps/doors
\end{itemize}