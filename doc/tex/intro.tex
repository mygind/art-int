\chapter{Introduction}

This report documents the work done in relation with project in DTU
course 02285 on ``Artificial Intelligence''.  This particular project
explores the use of AI (Artificial Intelligence) in the context of the
Sokoban game.

\section{Background and Rules}

Sokoban (means Warehouskeeper) was created in 1980 by Japanese
software company. The object of the game is to place boxes on goal
squares (their storage location).  Each level consists of a number of
goal squares and a corresponding number of boxes sorrounded by a
wall. This is called the floor plan (or board). To achieve the
objective, the user is represented with as a player able to move in
the directions: up, down, left and right. Boxes can not be pulled,
only pushed. \citep{cgw:sokoban}


Player and boxes can stand on goal squares. If a wall occupies a space
nothing else can be there and the wall cannot be moved from that
space.  The floor plan should be completely enclosed by a continuous
wall.

The game itself is rich in complexity and difficulty can vary from
very easy to incredibly complex, for both human players and automatic
solvers. This of course depends on the initial floor plan.

The high game complexity stems from, among other things: 
\begin{itemize}
\item The game is PSPACE Complete. \cite{culberson97sokobanpspace}
\item Some game states cannot be reversed (deadlocks, pushing boxes against walls, etc.)
\item Up to moves at each state. (Grows as $n^4$.)
\end{itemize}


Figure \ref{fig:soko-org-screen} shows the floor plan of the first
level original game.

\begin{figure}
  \centering
  \includegraphics[width=0.7\textwidth]{sokoban-screenshot}
  \caption{Screenshot of the original Sokoban game. From Wikipedia:
    \url{http://en.wikipedia.org/wiki/Sokoban}.}
  \label{fig:soko-org-screen}
\end{figure}

\section{Problem Formulation}

The assignment leaves us rather free to chose our focus, but there are
some minimum requirements we must meet. In this list of tasks we chose
some of them to focus on and others to address more lightly.

\section{Primary Focus of the Project}
We chose to put our primary focus in the project and the report on
implementating a solver for arbitrary Sokoban floor plans.

This means that large parts of this report will deal with the design
and implemenation of such a solver. This also implies that we to a
lesser degree address the more academic studies of complexity in the
game.

We will implement at least 3 different techniques for solving the
Sokoban challenges:
\begin{itemize}
\item Simple breadth first progression planner. %done
\item The same as above, only using a general heuristic. 
\item The same as above, replacing the general heuristic with one we design ourselves.
\end{itemize}

\section{Scope Limitation}
% EXPLAIN WHAT WE DID NOT DO. AND WHY!



